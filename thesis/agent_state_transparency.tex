\chapter{Internal agent state transparency}
% As previously shown, the TrustChain solution is a scalable, high-throughput, immutable append-only database. However, it is lacking state transparency  which makes sharing of information not enforceable. We can define an extended architecture which adds state transparency to TrustChain
% 1. We have shown that the fact that states of agents are not recorded on the blockchain adds all kinds of ambiguities about their behavior such that we cannot enforce rules in the system, as we cannot verify their behavior.
% 2. By recording the root hash of the Merkle tree of the database on each block, agents will be required to prove of which blocks their database exists. However, just this is not enough as from the hash itself no other agent can verify that the database is consistent with the data received by other agents. 
% 3. In order to solve this second problem, we have two options:
%     1. In each interaction in which blocks are transferred/exchanged, the sending party needs to sign a transfer block which includes the root hash of the Merkle tree of the blocks sent.
%     2. With each block transfer, both agents exchange all blocks such that they achieve the exact same state of the database (anti-entropy). In that case they don’t need to record the exchanged blocks but can each calculate the hash of the database and if they agree, they know that their databases are the same.
% 4. This makes available some powerful verification tools:
%     1. First of all, the addition publicly shows the knowledge of an agent. That means an agent cannot lie about not having some information or having information. Telling the truth about his information is strategy-proof.
%     2. We can replay the life-time of an agent and completely verify the behavior including application specific rules. 

In the previous chapters we analyzed the problem of information dissemination of data for a 
distributed repuation system built on the architecture of TrustChain. We have argued that in order
to make the dissemination strategy proof, which is neccessary to guarantee that no agent deviates 
from the desired behavior without being detected and properly dealt with, we need to publicly record
the action of sharing knowledge. In this chapter we propose an extension of the TrustChain architecture
that enables strategy proof information dissemination and validation in a distributed TrustChain 
based network without deminishing the scalability properties.

\section{Concept proposal}
In the previous chapter we formally defined the internal state of an agent as not only the encoutners
of agent's own encounters but the complete knowledge of the network the agent has. That knowledge 
can be represented by the set of encounters the agent is aware of. Based on this we can define a 
desired property that a fabric like TrustChain needs to fulfill in order to provide strategy-proof 
sharing of information.

\begin{defn}[Interal agent state transparency]
    The internal agent state is transparent, and therefore this property is fulfilled, iff:

    \begin{itemize}
        \item an agent $a_i$ can require an agent $a_j$'s internal state from any point in time
        \item an agent $a_i$ can determine whether an agent $a_j$ is lying about her internal state
    \end{itemize}
\end{defn}

When the property is fulfilled there should be an exchange protocol after which an agent is aware 
of all information the other agent has as well as a verfication protocol which validates that the 
claimed internal state is indeed valid and complete.

In order to achieve internal agent state transparency for TrustChain, any exchanged information
needs to be public. We will later describe the incentive for agents to actually perform
this exchange. So, conceptually a record exchange can be defined as follows.

\begin{defn}[Record exchange]
    A record exchange $X_k$ of size $m$ contains a list of transaction blocks $X_k = \{ B^{TX}_{k,0}, B^{TX}_{k,1}, \dots B^{TX}_{k,m}\}$.
\end{defn}

Finally, the complete state can be described by the union of performed transactions and transactions
obtained through exchagnes.

\begin{defn}[Internal state composition]
    The internal state $S_i$ of agent $a_i$ with true transaction history $D'_i$ and $k$ exchanges
    can be inferred as follows.

    $S_i = D'_i \bigcup \{ X_{i,0} \bigcup X_{i,1} \bigcup \dots X_{i,k} \}$
\end{defn}

\section{Implementation of internal agent state transparency in TrustChain}
\label{sec:implementation_state_transparency}
In the previous section we conceptually defined the internal agent state transparency for TrustChain.
This section will describe how those concepts can be implemented in a working distributed software
system. 

First and foremost, a new type of record needs to be created in which no application specific 
transactions, but system transactions of other records will be documented. We therefore extend the
TrustChain architecture (described in detail in {\color{red} WHERE?}) with exchange blocks 

\begin{defn}[Exchange blocks]
    An exchange block $B^{EX} = \lAngle \textit{ex}_{\text{up}}, \textit{ex}_{\text{down}}, \textit{pk}_i, \textit{seq}_i, \textit{pk}_j, \textit{seq}_j, \textit{sig}_i, \textit{sig}_j, \textit{hash}_{B(t-1)} \rAngle$ is defined as a tuple, where:
    
    \begin{itemize}
        \item $\textit{ex}_{\text{up}}$ is the top hash of the list of hashes of blocks the initiator $a_i$ shared with $a_j$
        \item $\textit{ex}_{\text{down}}$ is the top hash of the list of hashes of blocks the responder $a_j$ shared with $a_i$
        \item $\textit{pk}_i$ is the public key of the initiator of the exchange, agent $a_i$
        \item $\textit{seq}_i$ is the sequence number of the block in the history of interactions of agent $a_i$
        \item $\textit{pk}_j$ is the public key of the responder of the transactions, agent $a_j$
        \item $\textit{seq}_j$ is the sequence number of the block in the history of interactions of agent $a_j$
        \item $\textit{sig}_i$ is the signature by the initiating agent $a_i$
        \item $\textit{sig}_j$ is the cryptographic signature by the responding agent $a_j$
        \item $\textit{hash}_{B(t-1)}$ is the hash of the previous block
    \end{itemize}
\end{defn}

Each exchange block describes a pairwise exchange of blocks. The exchange is deliberately made 
bidirectional in order to provide an incentive for both agents to sign the exchange and keep to the 
promise of publishing the exchange on their chain. Instead of publishing the exact blocks exchanged,
a list of block hashes is created and the hash of that list is published in the exchange block. One 
hash for the block transferred from agents $a_i$ to $a_j$ and another hash for the blocks transferred in the 
other direction. This reduces the amount of data put directly on the chain, however it makes it 
impossible to infer the internal state of the agent from the chain only. That is why each agent 
internally needs to keep track of the actual content of the exchanges. 

\begin{defn}[Exchange storage map]
    Each agent keeps track of the content of all exchanges with an exchange storage map 
    $F: \textit{hash}_{B^{EX}(t)} -> X(t)$ which maps the hashes of exchange blocks to the respective record
    exchange, so the list of blocks the agent received in the exchange described by $B^{EX}(t)$.
\end{defn}

The combination of exchange root hashes and the exchange storage kept by each agent enables tamper-proof
exchange of information. This enables agents to determine the state of another agent. Consider an 
agent $a_i$ that tries to determine the state of agent $a_j$. Agent $a_i$ requests both the chain 
and the exchange map from agent $a_j$. If $a_j$ does not respond, agent $a_i$ is not able to determine 
the state and has to assume that $a_j$ does not respond in order to hide malicious behavior. In that case
$a_i$ will add $a_j$ to the list of blocked agents. 

\begin{algorithm}
\caption{Determining the internal state of another agent}\label{alg:internal_state}
\begin{algorithmic}[1]
\Procedure{determineState}{}
\State {$a_i: \text{Initiator}$}
\State {$a_j: \text{Responder}$}
\State {$\textit{chain} \gets \textit{request\_chain($a_j$)}$}
\State {$F \gets \textit{request\_exchanges($a_j$)}$}
\State {$\textit{state} \gets []$}
\If {$\text{not } \textit{chain } \text{or not } F$} \Return false
\EndIf
\ForAll{$\textit{B} \text{ in } \textit{chain}$ }
\If {$\textit{B} \text{ is an exchange}$}
\State{ $\textit{state} = \textit{state } \bigcup F(B)$ }
\EndIf
\State{ $\textit{state} = \textit{state } \bigcup \{ B\}$ }
\EndFor
\Return \textit{state}
\EndProcedure
\end{algorithmic}
\end{algorithm}

If both the chain and exchanges are available, the agent $a_i$ loops through the chain. Each block 
of the chain is added to the state and if the block is an exchange block, also all blocks that were
transmitted during the exchange documented by that block, which can be looked up in the exchange map
are added to the state. The result is the complete internal state of agent $a_j$. The algorithm is 
described formally in algorithm \ref{alg:internal_state}.

\section{Making sharing strategy proof}
\label{sec:strategy_proof_sharing}
{\color{red} How is strategy proof even defined?}
We have described how the internal agent state transparency property can be added to the TrustChain 
architecture, however this in in itself does not make dissemination of information strategy proof. 
It only creates a tool to document information exchanges in a tamper-proof manner. The next step is
to use the knowledge about information exchanges conducted by the agent to make decisions about 
future interactions. 

One solution is to build a system internal reputation which is based on the amount of data shared 
with other agents. Agents that exchange more data with other agents and initiate exchanges to obtain
more data should have increased reputation. Helping the system defend itself against attackers and 
making information widely available would then be advantageous as other agents see the good behavior,
have a better opinion of that agent and share data with them. 

Another solution is to define policies that require a certain amount of data exchange for each 
transactions. Let's say that the system designer introduces a policy which requires honest agents to
only interact with other agents that have at least one data exchange for each transaction they conduct.
This is simple to verify: the agent scans through the chain and counts the transaction and exchange
blocks and verifies that there are at least as many exchange as transaction blocks. Otherwise the 
agent will simply not interact with that agent.

\section{Possible ways to work around the security}
In this section a few examples of maliciously acting agents will be analyzed theoretically. The 
same will be done in experimental analysis with a proof-of-concept. We assume that all agents that 
interact with each other also verify each other's chain and internal state.

First of all, what the original TrustChain implementation allowed was to free-ride on information
sharing and verification. That is, even though honest nodes were polling other agents for information
on transactions and verified that no double-spends happend, other nodes were able to not do that and
still interact with honest agents. 

\begin{defn}[Lazy free-rider]
    We define a lazy free-rider as an agent that performs normal in transactions but never performs
    an exchange of information on transactions.
\end{defn}

In a network of honest agents, the lazy free-rider will quickly be found and blacklisted as that 
agent will simply not have any exchange blocks in the chain. 

A more elaborate way of free-riding is to exchange blocks and create exchanges but delete the data
as soon as it arrives. 

\begin{defn}[No storage free-rider]
    A no storage free-rider is an agent that does exchange data but does not store the received data.
\end{defn}

A no storage free-rider can only commit a few interactions before being detected as a fraud. Let's 
consider the first verification. The no storage free-rider agent only has the genesis block which 
is cheap to share. Therefore the first verification will go through and an exchange block will be 
created. However, the second verification is more difficult to pass as some information will have
been received as recorded in the exchange block but cannot be conjured on request. Any further 
verifier should therefore find that the agent is not able to provide the blocks of the first exchange
recorded on the agents chain. 

\section{Making verification strategy proof - Verification of verifiers}
Another way of free-riding on the original TrustChain architecture is to not actually validate any 
data. Even if we solve sharing of data, agents can still save computing power by not scanning their 
peers for malicious behavior in the hope that other agents on the network do perform the validations
and will inform them once they find something. The new extension makes it possible to detect this 
behavior in case they fail to report a malicious transaction. 

\begin{defn}[Validation free-rider]
    An agent that does perform honestly in terms of transactions and exchanges but does not validate
    the behavior peers is called a validation free-rider.
\end{defn}

The proposed extension of TrustChain enables an agent $a_i$ to determine the internal state of agent
$a_j$, that is obtain the list of all blocks of $a_j$. Consider that $a_j$ is a validation free-rider
and $a_i$ has obtained $a_j$'s internal state. If $a_j$ has obtained the knowledge of a malicious 
transactions, for example two conflicting blocks of a double-spend, $a_j$ will not have realized it 
as the validation has never been performed. However $a_i$ is aware of all blocks that $a_j$ has and 
therefore also aware that $a_j$ should be aware of the conflicting blocks if $a_j$ was honest. Now, 
$a_j$'s free-riding will be detected if $a_j$ has performed an interaction with the agent resopnsible
for the conflicting transactions. That is because any honest agent would have detected the conflict 
and ignored that agent except for colluters or validation free-riders. 

\section{Considerations of asynchronous transactions}
In the previous discussion we have actually simplified the system to one in which each agent only 
performs one transaction at a time. This makes it more straight-forward to understand the properties
and see that the extension does provide a strategy-proof dissemination. In a real-world system,
depending on the use case, a user might need to quickly engage in multiple transactions and 
communication delays do not allow to finish transactions first and then continue with the next. In 
previous work we have shown how TrustChain can work in an asynchronous way, by splitting the 
transaction blocks into a block proposal and a block agreement. This also creates additional 
complexity for the exchange blocks as the asynchronous reception of blocks needs to be recorded in
order to keep the property of internal agent state transparency.

\begin{defn}[Async exchange block]
\end{defn}

\section{}