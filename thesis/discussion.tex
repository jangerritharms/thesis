\chapter{Conclusion}
Creating trust between strangers on the internet is one of the most significant and challenging topics
of this century. Many of the current centralized reputation systems will in the long run not be able
to satisfy the demands for privacy, security and platform freedom. A distributed trust system can 
be a valid solution, yet it bears its own challenges. In this work we solved the problem of recording,
disseminating and verifying records of interactions which are the basis of any reputation or trust 
system. We will discuss in this last chapter our contribution to the design of a trust system and 
the future research towards a fully distributed, scalable and secure global trust system.

\section{Contributions}
Evidence of encounters are the core of any trust and reputation system. A reputation is little more
than a summary of a persons history of interactions in a certain context. Many positive interactions
lead to a good reputation whereas negative interactions lead to a bad reputation. The better someone's
reputation, the more we trust them and are willing to reciprocate their positive interactions. In a 
digital trust system the records of interactions therefore play a critical role in ensuring the 
correct value and validity of trust. Yet, in a distributed system without central government, no 
single point in the network has all information. This creates a problem for the verification and 
validation of interactions. 

\subsection{Model}
We first designed a model to study the problem theoretically. We model a network of agents. Agents 
can perform interactions and exchanges of information. An exchange can contain information about 
other interactions as well as other exchanges. We showed that without 
exchanging information, the network of agents is not able to defend against double spending attacks. In order to 
ensure that agents do gather and disseminate information, each exchange of information needs to be become part of an agents history
similar to interactions. Agents thus need to prove 
with their history that they did exchange information. Finally we show that if agents exchange all their 
information with every encounter, additional desired security properties can be ensured. This includes
that agents can repeat verification of their partner and thus be sure they did not perform any 
invalid interactions. 

\subsection{Architecture}
After proving theoretically the importance of exchanging information, we design an architecture that allows to 
record and verify transactions and exchanges of information. For that purpose we extend our
TrustChain fabric, which is a blockchain-based, scalability-focussed solution for recording 
interactions in distributed systems. The entangled hashchain approach of TrustChain creates advanced
tools for the detection of manipulation attacks. However TrustChain's security greatly relies on the
willingness of agents to obtain their peers blocks and verify them. 

Our extension adds exchange blocks to TrustChain which record any block exchanges between agents. We call
this exchange transparency.
Exchange blocks contain a hash of the blocks that were exchanged and they are recorded on the chains
in the same way as normal transactions. This creates for each agent a tamper-proof history of exchanges and
transactions. By recording a signed hash of each exchange, agents can be asked to provide the exchanged
blocks such that the hash can be recalculated by the verifier. Failure to provide the correct block that
match the exchange hashes will lead to mistrust. This allows honest agents to distinguish
between honest agents that also exchange data, free-riders that do not record exchanges or only empty
exchanges and manipulators that have invalid data on their chain. In applications that require the 
highest level of security, agents can be required to exchange all knowledge. This allows any agent 
to redo any verification in the subject's history. If all verifications pass, the agent acted exactly
like the verifier would do and thus can be considered very trustworthy.

\subsection{Implementation}
Our final contribution is an implementation of the architecture and verification mechanism as well as
an experimental analysis of its manipulation resistance. In our experiments small groups of agents 
are emulated. The groups consist of honest agents that implement the exchange and verification 
mechanism according to design, as well as free-riders and manipulators. We show with two examples
that honest agents are able to detect manipulating agents. This is made possible by the TrustChain 
architecture and is to be expected. But it was not possible yet to also defend against lazy agents
that do not help in detecting those agents. Our extended architecture allows honest agents to 
detect exchange free-riders, even if they are colluding and also agents that knowingly interact with
malicious nodes. As a consequence, any agent that wants to interact with an honest agent in the future
needs to exchange data with their partners and make sure that they are not in any way malicious.

We conclude that exchange transparency ensures that agents acquire and distribute knowledge. Our 
extension of TrustChain is flexible to allow different levels of information dissemination. It allows
agents to replay the history of another agent to validate their complete historic behavior. As such
our work is a major step in securing our future distributed trust system.

\subsection{Research question}
With respect to our research question we can make the following conclusions. Recording exchanges 
leads to a strong incentive to disseminate and verify encounters between agents. This way honesty
of encounter reports can be ensured as well as consistency. Therefore our architecture could be 
seen as a possible solution to problem of recording encounters in a trust system. 

However, our system creates restrictions for the scalability of the system. We require much more 
storage than only the own transactions. Also, extensive verifications can become very costly once 
agents' chain become significant in size. Therefore, this system is only partly applicable in the 
context of a global-scale trust system. Other innovations, like a system-level trust system could 
help lessen the scalability issues.

\section{Future work}
Looking ahead, there are still major challenges to achieve our ambitious goal of building the 
internet of trust. On the one hand, we have created a prototype for exchange transparency but some 
challenges remain and need to be researched. On the other hand multiple other large challenges need
to be solved like a strong identity system and the Sybil attack.

\subsection{Further development of exchange mechanism}
The recording of exchanges has great influence on many components of the trust system. Not all of 
these could be analyzed in this work. 

First of all a larger study on the scalability of our mechanism
needs to be done. Redoing the verification of an agents complete history is very work intensive once the 
history grows. It is possible to do this in multiple steps, each time two agents interact they only
need to verify the history that is changed. Also, the Network-State-Exchange policy will lead to 
storing all data of the network, this can in many cases not be acceptable. Therefore a mechanism 
needs to be designed to delete data without impacting the security of the system to badly. 

Another interesting direction for further research is building a system-level trust system which we
mentioned in Section~\ref{sec:system_trust}. Instead of performing complete verification and exchanges
with every interaction, agents build system level trust. After a few exchanges, verifications and 
succesful interactions, two agents can trust each other's verifications. Thus if one of the two 
agents verifies a third, the other agent believes this verification is correct and does not perform
it again. Instead of stopping to verify trusted agents, verifications can become less frequent. Any 
misbehavior destroys the trust. It should be noted that the system-level trust is different from the
trust in the application context: an agent can only be downloading but still always share and verify
transaction data. Such a node will have a bad application context reputation but a great system-level
reputation.

\subsection{Future development of trust system}
In Chapter \ref{chap:problem} we have introduced the architecture of a trust system. We have tackled
in this work the mechanism for recording transactions and the distribution of those records. But
other challenges remain. 

A strong identity system which relates the public keys to true identities through government issued
documents can greatly improve security. As we have discussed in Section~\ref{sec:attacks} some 
attacks like whitewashing and Sybil attacks are possible because identities are cheap. If on the 
other hand identities are coupled with real-world, single instance values or documents such as 
passports, it becomes much harder to renew an identity. A downside of such an approach is that 
it creates a dependency on external documents. In countries where governments are corrupt or 
unable to act such documents might not be obtainable. In that case biometric data could help to 
create strong identities. 

Another way we envision to create a stronger protection for our trust system is to use latency. The 
connection latency between nodes in a network is a physical value that defines a neighborhood for  
each node. If we make trust dependent on a low latency we believe that attacks will be much harder.
An attack usually requires an interaction, so some initial trust. However trust can only be obtained
from inside a physical distance. Thus attackers need to be able to access machines that are close
to their targets. Attacks from distant governments or hacker groups require infiltration of local
machines, putting another obstacle in their way.

% In this work we presented a strategy-proof mechanism for information dissemination. Applied to our 
% distributed blockchain based trust system we are able to effectively defend against dissemination 
% and verification free-riders. It creates an incentive for each agent on the network to help defend 
% the network against any lazy or malicious behavior. It thereby is a major step towards a secure, 
% distributed and scalable trust system.

% * we defined a new blockchain system based on TrustChain which provides internal agent state 
% transparency/gossip transparency
% * we formally proof that the architecture provides a complete view of the internal state of the 
% agent
% * we defined a specific mechanism that makes use of the archiecture
% * we experimentally proof that honest agents are able to eventually identify free-riders and malicious
% agents

% \section{Future research}

% \subsection{Further developing this mechanism}
% * incremental 
% * research scalability properties for this mechanism
% * locality by interacting with those that have similar information
% * sybil attack resistance by checking that new agents paid their dues

% \subsection{Next steps for the trust system}
% * locality with ping
% * strong identity system