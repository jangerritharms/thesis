\chapter*{Abstract}
\setheader{Abstract}

Trust on the internet is largely facilitated by reputation systems on centralized online platforms. 
However reports of data breaches and privacy issues on such platforms are getting more frequent.
We argue that only a decentralized trust system can enable a privacy-driven and fair 
future of the online economy. This requires a scalable system to record interactions and ensure
the dissemination and consistency of records. We propose a mechanism that incentivizes agents to 
broadcast and verify each others interaction records. The underlying architecture is TrustChain, 
a pairwise ledger designed for scalable recording transactions. In TrustChain each node records 
their transactions on a personal ledger. We extend this ledger with the recording of block 
exchanges. By making past information exchanges transparent to other agents the knowledge state of 
each agent is public. This allows to discriminate based on the exchange behavior of agents. Also, it
leads agents to verify potential partners as transactions with knowingly malicious users leads to 
proof-of-fraud. We formally analyze the recording of exchanges and show that free-riding nodes that do
not exchange or verify can be detected. The results are confirmed with experiments on an open-source 
implementation that we provide.

\begin{flushright}
{\makeatletter\itshape
    \@author \\
    Delft, August 2018
\makeatother}
\end{flushright}

