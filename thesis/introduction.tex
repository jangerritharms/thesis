\chapter{Introduction}
The development of the personal computer and the internet as the digital infrastructure for global
connectivity drastically impacted almost every feature of our lives. Many cannot imagine life 
without constant global acccess to knowledge, photos, social contacts and many more things. When
Tim Berners-Lee developed the World Wide Web in the 1980s he imagined it as a globally distributed
source of knowledge in form of a network where each node is equal, can both contribute and consume
knowledge in order to improve cooperation and sharing of information. The huge success of the 
internet proved his idea right but at the same time led to problems: with global connectivity the
openness of the internet's network invited malicious users. Also some services offered on the web 
were highly requested which required a highly advanced infrastructure to handle such traffic. 
Reasons such as these led to the centralization of the internet and the monopolization of services.

A similar trend can be found in many aspects of modern society: global production has replaced
local production, global retailers replaced local retailers and mega cities replace rural villages.
The strive for efficiency, growth and success leads to entities combining forces and creating ever
stronger super-entities that improve processes to become more powerful. Mostly centralization has
led to widespread welfare through more jobs, more affordable products and more efficient distribution
of goods and services. The same is true for the internet: facebook has connected more than 2 billion
people, google allows you to find information on anything and youtube is delivering entertainment;
all for free, available to anyone willing to create an account, world wide. 

But not everything about centralization is advanatgeous. Companies in powerful monopoly-like market
positions can misuse their importance to influence politicians to act in their favor. German 
automobile manufacturers for example lobby against policies for reducing emissions of cars in 
European Parliament or speed-limits on the German Autobahn. In the digital world, Facebook misuses
their user's data and allows political propaganda to be spread to millions of people. Furthermore,
centralized application are prone to attacks by hackers as the servers are a single point of 
failure, and one data breach leads to the exposure of millions or even billions of users. Many
such examples have proven in the past that centralized applications are not safe.

Banks are another example of such centralized systems that have their client's trust but are proven
untrustworthy like during the financial crisis. This led to the development of Bitcoin, the first
decentralized, digital currency which works completely without central authorities and is safe 
against double-spending, an attack on digital financial systems in which two confilcting 
transactions are submitted, resulting in spending some amount of currency twice. Bitcoin uses a 
single global hash chain of blocks (sets) of transactions. The hash chain leads to a chronology 
of transactions which makes double-spending impossible. All nodes on the Bitcoin network agree on
this sequence of transactions using an algorithm called Proof-of-work(PoW) which solves a hard
mathematical problem, expending time and resources. The first to solve the problem can publish a
new block on the chain including a new (confirmed) set of transactions. Bitcoin is very secure, 
completely decentralized but it's Achilles heel is scalability: with a single global chain of 
transactions and the current block time of 10 minutes the theoretical throughput is 7 transactions
per second. 

Since the advent of Bitcoin many other digital currencies have appeard and some have found almost
similar praise as Bitcoin, however the scalability problem still exists with most of the currencies.
Vitalik Buterin, co-founder of the second largest digital currency network Ethereum, describes
this as a trilemma: of the three desireable properties decentralization, scalability and security,
at most two can be attained by one blockchain system at the same time. But in order to be usable 
for a global currency or as a global transaction storage, scalability will be neccessary.

This master thesis is concerned with TrustChain, a blockchain system developed at the Blockchain 
Lab at TU Delft. TrustChain has no global consensus on all transactions which removes the 
bottleneck for transaction throughput. Instead of a single blockchain for all transactions each 
entity on the network has their own blockchain. We have shown in previous work that this system 
achieves horizontal scalability, so additional nodes on the network lead to additional throughput.
The scalability comes at the cost of security guarantees, most prominently the double spending 
attack and the sybil attack. The double spending attack is easy to perform because an agent can 
simply publish a second conflicting transaction with another node without sharing the original 
transaction. An attack is called a Sybil attack when an adversary creates many fake entities on the
network to obtain a large part of the voting power or resources. This attack is usually prevented 
through verification of the agent, for example bitcoin verifies nodes by letting them perform work.
Trustchain is built in a way that these attacks are not prevented, which is too costly to be 
scalable, but instead are detectable. Both transactions of a double-spend attack are stored on the
blockchains of the two exploited agents. By spreading the records of these transactions in the 
network, eventually a node will have both versions of the same transaction and will identify the
attacking node.

Next to the research on Trustchain the Blockchain Lab is concerned with real-world testing of a
deployed system. The platform for those tests is Tribler, an anonymized onion-routing protected
bittorrent client. A common problem with the bittorrent network is that it is not inherently 
protected against free-riders as there is a social dilemma in sharing resources with other agents
on the network. Uploading data to other nodes is costly in terms of bandwidth without any direct 
reward, but if noone decides to upload nobody can download and the system breaks. This dilemma is
a form of what is known as the Prisoner's dilemma in classical game theory. With BarterCast our 
research group has devleoped a system to prevent free-riding in the bittorrent network and recently
this mechanism has been improved using Trustchain as a tamper-proof way to store transaction records.
This work will also use the Tribler example as context.

Our contribution will be targeted at the dissemination mechanism of transaction records on the 
Trustchain network. Because every node has it's own chain, each node has only a subset of the data.
This is different from blockchain systems with global consensus in which all agents act on the same
data. This has considerable influence on the attack defence and accounting mechanisms.
While Trustchain is made in a way that makes attacks detectable this is only possible if 
an agent collects data from other agents on the network. Thus, the dissemination and validation of
data is of very high importance for the proper functioning of thu system. 
This mechanism has not yet been formerly been defined and it's implications for security and 
scalability of the Trustchain have not been researched in depth. Specifically this work 
contributes:

\begin{itemize}
    \item We formally introduce \textit{Pairwise endorsements} -- a dissemination mechanism that 
    records information exchange and validation on-chain, making information sharing strategy-proof
    and information tamper-proof
    \item We analyze the security, scalability and correctness of the dissemination mechanism
    \item We provide an implementation and experiment results to show the working of the mechanism.
\end{itemize}

The rest of this report will be structured as follows. In the next chapter, the problem will be 
discussed in more detail.
