\chapter{Introduction}
Eventhough humans are most commonly modelled as rational entities and cooperation
is not a rational strategy, it has often been observed throughout history~\cite{Axelrod1390}.
Trust is the basis for any transaction between humans, be it trading or services. 
The handshake to settle on the terms 
of a transaction is a sign of respect, cooperation and trust. Once a service is
provided by one agent, a trustworthy second agent in the transaction pays the debt 
or runs the risk of loosing his reputation, which worsens the terms of following 
transactions or even prevents him from further transactions.

Over time, the way we perform transactions has changed. In the pre-industrial age,
transactions where most common between people of the same small community with some traders
travelling between communities, however transaction always happened with direct 
personal contact. In the post-industrial age, large multi-national companies enabled transactions
and trade on a global scale. The most prevalent type of transaction were asymmetric transactions between
customer and business. Customers trusted the company to provide high quality service 
for the given money without exploiting the flaws of the system. Lately, the internet enabled
peer-to-peer transactions via a trusted third party, also called the ``sharing economy.''
The most well-known examples of this are Airbnb and Uber. They provide the infrastructure to connect strangers with 
matching offer and demand, leading to direct transactions between customers. Trust in the latter
example is two-fold: a reputation system establishes trust between customers and both
parties trust in the platform to solve conflicts should they arise.

However, both multi-national companies and the sharing economy have flaws. Large multi-national
companies can get very powerful to the point where they are hard to control, giving them
the freedom to change the rules or cheat the system. Companies can acquire a monopoly 
position or multiple companies can agree on prices, leading to a artificially high price
for the product they provide. Also, companies can pressure smaller sub-contractors to 
produce at very low cost, which leads to bad working conditions. The sharing economy offers
a market place which directly connects service provider and customer for a small fee. The
problem here is a very different one. The reputation system is locked to the specific platform, 
a trusted user of Airbnb cannot use the reputation on any other platform. Thus the platform 
is the owner of the reputation, not the service provider. Even, between different platforms
that offer the same service like Uber and Lyft, driver profiles are not sharable. Also their
market is fragmented, meaning some cities might be served better by one service but if you
use the other service you are excluded from getting a ride. The discussion leads to the 
conclusion that what is missing is a global, fair and open marketplace 
in which relative strangers can reliably exchange goods and services, without trusting 
intermediators with either their data or stake. 

Bitcoin was offered as the answer to the above request. Trust is replaced with cryptographic
proof enabling the transaction of money with extremely low probability of fraud or tampering. 
The high level of security comes at the cost of efficiency and speed. The processing of 
transaction through creating the proof-of-work consensus is costly in terms of resources. In 
2017 the processing of transaction was using as much energy as all of Denmark. This also 
leads to high transaction costs. At the same time Bitcoin has the infamous restriction of
a maximum of 7 transactions per second, which is far from the throughput of convetional bank
transaction systems. Many other crypto currencies have been developed (Ethereum, Ripple, 
Stellar Lumens, etc) as well as off-chain transaction systems such as the Lightning network 
have been proposed. However no cryptocurrency has proven horizontal scalability in practice
as the time of writing.

Trustchain, developed by the Blockchain lab which is associated with the TU Delft,
is a proposal for a scalable, multi-chain fabric. It takes an approach that is closer to the equivalent
of non-digital transactions. Trust are not replaced with costly calculations but instead made 
the basic currency in the network. Security is achieved through long-lived identities and reputation
which is built over time through succesful interactions. Reputation built in one context is then 
influencing other contexts in the future and identities increase their value through continous
good behavior. Misbehaving on the other hand leads to the loss of reputation and consequently to
worse terms in transactions in the extreme case, begin outcast. Transparency and consequences breed 
honesty, and honesty breeds trust. 

In previous work we have presented the Multichain architecture of TrustChain. It enables horizontal scalability, full decentralization
and reliable transactions for any type of application. Also, the concept of implicit consensus was
introduced in which global consensus is established on checkpoints instead of all transactions. 
This does not prevent double spending but instead makes it discoverable after it happend. This
way, before performing a transaction we can check the transaction history of the other party 
see whether their chain turns out to be valid. The validity of a chain and succesful interactions build trust in the 
node's intention to act correctly. However, a valid transaction history does not 
always mean that the transactions were actually meaningful. One famous attack is the Sybil-attack in which a node creates many 
sybils with which is performs valid interactions in order to boost the reputation of one node. 
Once a reputation has been established, the attacker is able to exploit people in the network
if they think the reputation is high enough to take the risk of a transaction.

Therefore, we have introduced Peronsalized Temporal PageRank, an algorithm that determines a 
ranking of trustworthiness of a node's peers and their peers in a sybil-resistant fashion. Yet, 
this ranking is only local and reputation is only useful for building trust once it is shared 
with others. Therefore a mechanism for distributing the subjective trustrankings and their
aggregation is crucial for the success of a global trustworthiness ranking. This will be the 
goal of this thesis. 

The rest of this thesis is organized as follows. ...