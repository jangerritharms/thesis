\chapter{Related work}
In the previous two chapters we have shown that a need exists for a decentralized accounting system 
in order to create a global infrastructure for secure, anonymous digital transactions that does not 
require control through a trusted third party. This need has been identified before and work has 
been performed both in the scientific community as well as the industry. In this chapter we will 
summarize those efforts, describe the short-comings of those approaches and define a basis for the 
work performed in this work.

\section{Applications of decentralized accounting systems}
The general concept of accounting is quite old as it is simply a recording of transactions between
two or more parties. Before the digital age those recordings were simply written text on paper, 
nowadays those recordings are stored in databases. We are concerned with another type, namely 
decentralized accounting systems. We identified three types of applications for decentralized 
accounting systems: cryptocurrencies, distributed work systems and reputation systems. 

\subsection{Cryptocurrencies}
In the years 2007 and 2008 the global financial crisis shattered the global economy, lead to many
people loosing house and job and diminished the trust clients had in banks to keep their money safe.
Politics discussed the problem and proposed to regulate the banks more but with little impact. 
However something else promised to change the banking world: the first white-paper for a 
decentralized digital currency without any need for a trusted third party, Bitcoin, was announced. 

\subsubsection{Bitcoin}
Before the announcement of Bitcoin it was assumed that in order to verify the correctness of 
transactions between parties and prevent cheating with digital money a bank or credit card company
was needed. Bitcoin proved them wrong by creating a hash-based chain of transaction blocks, a global 
ledger, that is shared among all users of the network. The acceptance of transactions is managed by 
a process called ``mining'' which ensures that only the majority of CPU power can publish new 
block. A blocks contains a fixed number of transactions and the Bitcoin network makes sure that a
block is created once every 10 minutes. All mining node will execute the proof-of-work mechanism: 
in order to publish a block a value needs to be found that, when hashed with a certain hashing 
function like SHA-256, starts with a certain number of zeros. Depending on how many CPUs are active
on the network the problem can be increased in difficulty by requiring more zeros at the beginning 
of the hashed value. Once a new block is published other nodes will validate the transactions and 
if they agree, will show their acceptance by working on creating the next block. This system ensures
that as long as a majority of CPU power is owned by honest nodes, they will outpace the rest of the
network in solving the hashing puzzle and creating valid blocks. Nodes will accept the longest chain
and the transactions will be valid.

The Bitcoin approach solved many problems assuming that an honest majority exists: first and 
foremost the double-spending of funds is prevented because the Bitcoin blockchain creates one global
order of valid transactions. Also spamming 

\subsection{Reputation systems}


\paragraph{Centralized} ...

\paragraph{Decentralized} ...

\subsection{Applications}

\paragraph{Market}
\paragraph{Sharing economy}
\paragraph{BitTorrent}
\paragraph{Tribler}

\section{TrustChain}

\subsection{Data structure}
\subsection{Accounting mechanism}
\paragraph{Definition of trust and reputation}
\subsection{Subjective graph}
\subsection{Consensus}

