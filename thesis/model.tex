\chapter{Related work}

\section{Context}

\paragraph{To understand the relevance} of this work we need to put it into the
perspective of the context of reputation systems and their applications. 
With the wide-spread use of the internet for trade, sharing and communicating,
interactions between strangers living far apart are wide-spread. Many applications
allow for exploitation through manipulation and taking advantage of the asymmetry
of information. For example a buyer needs to pay for products before even seeing
and afer receiving the money the seller actually does not have any incentive to
deliver a product. This opens the door for exploitation. In order to solve this
problem a reputation system is put into place in ebay, which shows publicly what
other buyers said about the seller in the past. If the seller has not delivered
a product a few times, the reputation will tainted with negative reviews and 
buyers will be reluctant to interact with this seller in the future. Also, having
no reputation at all will seem suspicious and buyers will have little trust in
the seller. This influences the prices of products that this seller can ask for.

\subsection{Reputation systems}

Reputation is a concept that not only exists on internet platforms, but it is 
an important part of everyday life. Everyone has opinions about friends, colleagues,
companies, newspapers, weather forecasts and many more things. Reputation and 
trust are subjective quantities, we are influenced by gossip from our peers. If
all of our friends tell us that a certain company makes crappy laptops, we are
probably choosing for a different companies laptop.

In this work we will focus on reputation systems for internet platforms. In this
category there are two different approaches, the centralized approach and the 
decentralized approach.

\paragraph{Centralized} ...

\paragraph{Decentralized} ...

\subsection{Applications}

\paragraph{Market}
\paragraph{Sharing economy}
\paragraph{BitTorrent}
\paragraph{Tribler}

\section{TrustChain}

\subsection{Data structure}
\subsection{Accounting mechanism}
\paragraph{Definition of trust and reputation}
\subsection{Subjective graph}
\subsection{Consensus}

