\chapter{Formal model}
We aim to show in this section the importance of dissemination and verification of interactions 
records. This is done by defining a model within which we can proof with rigorous math that record
dissemination and verification are essential for securing the trust system. Also we define an 
extension to our basic model which allows to protect the network against free-riders which 
jeopardize the overall security. Our proposal is to introduce gossip transparency which can 
effectively punish free-riders that do not acquire and disseminate interaction records. We further 
show that it is possible to detect agents that consciously interact with malicious agents, exposing
them as verification free-riders or accomplice.

\section{Basic model}
We make use of the ordered interaction model which has been introduced in previous work from 
our group in \cite{OTTE2017}. 

\begin{defn}[Ordered interaction model]. An ordered interaction model $M = \langle P, I, a, w \rangle$
    consists of two sets and two functions.
    \begin{itemize}
        \item $P$, a finite set of agents
        \item $I$, a finite set of interactions
        \item $a : I \rightarrow P \times P$, a function mapping each interaction to the agents involved in it.
        \item $w : I \times P \rightarrow \mathbb{R}_{\geq0}$, a function which describes the contribution of an agent in an interaction
    \end{itemize}
\end{defn}

The interactions that one agent is involved in can be represented by the personal interaction history.

\[ I_p = \{ i \in I : p \in a(i) \}\]

The set $I_p$ is strictly totally ordered by $<$. Let $i_p(t)$ be the $t$th interaction of agent 
$p$. The we can alternatively write the set of $I_p = \{ i_p(1), i_p(2), ... i_p(n) \}$.
An interaction happen between two agents, therefore only those two agents observe the interaction.
The set of observed interactions for the ordered interaction model is equal to the personal 
interactions of the agent. The observations help to define the knowledge that one agent has about
another. 

\begin{defn}[Observed peer interactions] 
    \label{def:peer_history}
    An agent $q$ with personal $I_q$ observes only a partial history
    of its peers equal to their direct interactions.
    \[ I_{q,p} = \{ i \in I : p \in a(i), q \in a(i) \}\]
\end{defn}

\begin{defn}[Reputation function] 
    A reputation function $R$ takes as input an observed peer history $I_{p,q}$ and determines the score $S^R_{p,q}(I_{p,q}) \in \mathbb{R}$.
\end{defn}

The fact that interactions are only observed by the interacting agents makes the system vulnerable
against attacks. We can define a fork and double spend attack. 

\begin{defn}[Fork and double spend]. A malicious agent $p$ can have two conflicting interactions 
    $i \in I$ and $j \in I$ at the exact same time. Both interactions have a different partner $a(i) = \{p, q\}$ 
    and $a(j) = \{p, r\}$ Accordingly, the personal history of $p$ $I_p = \{i_p(1) ... i_p(t), i_p(t)', i_p(t+1), ... i_p(n)\}$ is not a totally ordered set anymore.
    The attack can be detected if for any agent $s$ with observed peer history of peer $p$ both 
    $i_p(t) \in I_{s,p}$ and $i_p(t)' \in I_{s,p}$.
\end{defn}

\begin{thm}[Detectability of fork] 
    The fork and double spend cannot be detected if agents only observe their own interactions.
\end{thm}
\begin{proof}
    Let $p$ be the attacking agent and let $i$ and $j$ be the conflicting interactions of the attacks.
    Further, let $q$ be the partner in interactions $i$ such that $a(i) = \{p, q\}$ and let $r$ be
    the partner of interaction $j$ such that $a(j) = \{p, r\}$. According to the definition 
    \ref{def:peer_history} $i \in I_{q, p}, j \notin I_{q, p}$ and $i \notin I_{r, p}, j \in I_{r, p}$.
    Moreover, all other agents $s \in P$ are aware of neither $i$ or $j$ because only direct 
    interactions are observed.
\end{proof}

We have shown that the current model of the network does not offer the possibility for agents to 
detect a fork. This is a major security issue which needs to be resolved. 

\section{Extended model}
In order to defend against the double spend attack, agents need to observe more interactions. We 
therefore extend the model with \textit{exchanges} which allow agents to communicate their knowledge.
This process is generally known as gossiping. {\color{red} cite some papers that discuss gossiping}
We further make the assumption that the network is static, so no agents leave or join the network.

\begin{defn}[Ordered interaction and exchange model]. An ordered interaction model $O = \langle M, I, b, x \rangle$
    consists of two sets and two functions. 
    \begin{itemize}
        \item $M$, an ordered interaction model
        \item $E$, a finite set of exchanges
        \item $b : E \rightarrow P \times P$, a function mapping each exchange to the agents involved in it.
        \item $x : E \times P \rightarrow I^+$, a function which describes the interactions that an agent received in an exchange. $+$ refers to the Kleene plus.
    \end{itemize}

    We assume that agents choose a random peer with uniform probability for an exchange and 
    exchange their complete history.
\end{defn}

Then we can redefine the knowledge that an agent has about its peers as follows.

\begin{defn}[Observed peer interactions] 
    \label{def:peer_history}
    Let $O$ be an ordered interaction and exchange model. An agent $q$ with personal history $I_q$ 
    observes the personal history of its peer $p$. The observed history is equal to their direct 
    interaction and $p$'s interactions that $q$ received from other peers through exchanges.
    \[ I_{q,p} = \{ i \in I : p \in a(i), q \in a(i) \} \cup \{ j \in I : e \in E, q \in b(e), j \in x(e, q), p \in a(j)\}\]
\end{defn}

\begin{thm}[Detectability of fork in extended model] 
    The fork and double spend will eventually be detected in the ordered interaction and exchange 
    model.
\end{thm}
\begin{proof}
    Let $p$ be the attacking agent and let $i$ and $j$ be the conflicting interactions of the 
    attacks. Further, let $q$ be the partner in interactions $i$ such that $a(i) = \{p, q\}$ and let 
    $r$ be the partner of interaction $j$ such that $a(j) = \{p, r\}$. Due to uniform random peer selection
    eventually $q$ will choose $r$ for an exchange, or the opposite. In that case $q$ will receive 
    interaction $j$ and $r$ will receive interaction $i$. In that case $q$ and $r$ fullfil 
    the conditions $i \in I_{q,p}$ and $j \in I_{r,p}$, respectively, for detection of the attack.
\end{proof}

We find that the exchange and verification of data is essential for defending the network against 
the double spend attack. However, if we assume that the exchange and verification of data creates a
cost for the agent, an agent might decide to not perform those activities. This jeopardizes the 
security of the network.

\begin{defn}[Dissemination free-rider]
    Given an Ordered interaction and exchange model $O$ a dissemination free-rider $p$ is an agent 
    who does not exchange data. Specifically, $E_p = \{ e \in E : p \in b(e) \} = \emptyset$. 
\end{defn}

\begin{thm}[Reputation of dissemination free-rider] 
    A dissemination free-rider in the ordered interaction and exchange model does not loose 
    reputation with respect to an honest agent.
\end{thm}
\begin{proof}
    The reputation function takes as input the observed peer history which does not include any encounters $e \in E$. Two agents $q$ and $r$ with a similar personal history $I_p$ and $I_q$ can
    therefore have the same score if evaluated by an agent $p$, $S^R_{p,q}(I_{p,q}) = S^R_{p,q}(I_{p,q}), E_q = \emptyset, E_p \neq \emptyset$. Thus, a dissemination free-rider does not loose any
    reputation.
\end{proof}


\section{Gossip transparency}
From the previous discussion it should be clear that dissemination of interaction records is an 
essential part of the security of the global trust system. Yet without recording and spreading the 
knowledge about exchanges it is possible to free-ride without contribution to the spreading of 
information without any repercussions. In this section we define the concept of gossip transparency.
By recording the sharing behavior in a similar way as interactions, agents are able to determine 
each others reputation based on both interactions and exchanges. 

The problem that we identified in the previous section is that although exchanges are a solution to
defending against the double spending attack we are not able to provide an incentive for agents to 
reliably perform exchanges and verify the received data. We now explore a model that does allow this
by recording and disseminating exchanges in a similar way as interactions. We introduce the concept
of an encounter between two agents in which both agents either perform an interaction of exchange 
of records.

\begin{defn}[Ordered encounter model]
    An ordered encounter model $M = \langle P, N, c, \tau, \omega, \chi \rangle$ consists of two 
    sets and four functions.
    \begin{itemize}
        \item $P$ a finite set of agents
        \item $N$ a finite set of encounters
        \item $c : N \rightarrow P \times P$ a functions mapping the encounter to the two agents involved
        \item $\tau : N \rightarrow \{\textit{interaction}, \textit{exchange}\}$ a function mapping 
        the encounters to the two possible types of encounters, interactions and exchanges
        \item $\omega : N \times P \rightarrow \mathbb{R}_{\geq0}$ a function describing the contribution
        of agents in interaction encounters
        \item $\chi : N \times P \rightarrow N^*$ a function describing the encounter information an 
        agent received in an exchange encounter, where $*$ is the Kleene star
    \end{itemize}
\end{defn}

The encounters that one agent is involved in can be represented by a totally ordered set.

\[ N_p = \{ n \in N : p \in c(n) \}\]

The model allows to observe exchanges as well as interactions. We can define the observed encounters
as follows.

\begin{defn}[Observed peer encounters] 
    \label{def:peer_history}
    Let $O$ be an ordered encounter model. An agent $q$ with personal history $N_q$ 
    observes the personal history of its peer $p$. The observed history is equal to their direct 
    interaction and $p$'s interactions that $q$ received from other peers through exchanges.
    \[ N_{q,p} = \{ n \in N : p \in c(n), q \in c(n) \} \cup \{ m \in N : \exists e \in N, q \in c(e), m \in \chi(e, q), p \in c(m)\}\]
\end{defn}

Finally, the reputation function should act on the observed peer encounters instead of the previously
defined observed peer interactions.

\begin{thm}[Detection of free-riders]
    In an ordered encounter model dissemination free-riders can be detected and penalized through 
    a reputation function.
\end{thm}

\begin{defn}[Verification free-rider]
    Given an Ordered interaction and exchange model $O$ a verification free-rider $p$ is an agent 
    that exchanges data but does not check whether that data is valid. 
\end{defn}


{\color{red}
\begin{itemize}
    \item Explain that because the reputation function only takes the interactions into account the
    it's possible to not exchange data and therefore free-ride 
    \item Also it's possible to not perform verification as a collaborator or free-rider
    \item We need to define another model in which interactions and exchanges are both recorded
\end{itemize}}