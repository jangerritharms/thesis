\chapter{Problem description}
In the introduction we make a case for the decentralization of applications that handle private
information or resources and argues that scalability is one of the main problems of the promising
blockchain technology to make such systems a reality. Trustchain is an approach that removes the 
main bottleneck that restricts the scalability of the most common blockchain fabrics, namely global
consensus. However the lack of agreement on a single accepted set of transactions has many 
implications for attack resistance and correctness guarantess of the system. This chapter 
introduces these implications and defines the problem that this work is supposed to tackle.

\section{Attacks}

\subsection{Double-spend attack}
One of the most challenging attacks that exist in distributed systems is the \textit{double-spend}
attack in which an adversary creates two conflicting transactions with two different agents without
telling each about the other, effectively using resources twice. In centralized systems this attack
is prevented by the central server which processed transactions in order and realizes that the 
resources were spent in the first version of the transaction. Bitcoin was the first decentralized
accounting system that solved this problem without a central, trusted entity. However the mining
which creates a single accepted sequence for transactions is costly in terms of time and resources.
Without global consensus Trustchain (discussed in more detail in section 3) is not able to prevent 
the double-spend attack. Instead, the double-spend attack will be recorded and therefore made 
detectable. The attacker sends two conflicting transactions to two different agents and keeps one,
but both partners write the conflicting blocks on their chains. If those two agents share their 
blocks with each other or both share their blocks with a third agent, the attack becomes detectable
because the two blocks are conflicting. The prevention of this attack therefore requires 
dissemination of transaction data across the network and constant checking for conflicting 
transactions by all agents.

\subsection{Sybil attack}
During a sybil attack an adversary takes control over many entities at the same time without making
this known to the network. The attacker can then use those entities to gain influence without any
real cost because the controlled entities can create proof of transactions without actually 
performing them.

This problem is very hard to detect because controlled entities can look like real agents to 
external observers. In centralized systems this is often prevented by requiring multiple 
authentication steps, for example scanning an identity card. Also if the creation of new agents has
some costs, the adversary needs to evaluate the possible advantage against the cost of creating
multiple agents.
In Bitcoin and other proof-of-work based cryptocurrencies the attack is avoided because the power
to create a new block is proportional to computational power, so whether the computational power
is spread over multiple agents or not does not matter to the voting power in the system.

For other decentralized systems the sybil attack continues to be a challenging problem. Many 
solutions have been proposed which analyze the topology of the network. Also an initial negative 
balance has been proposed by some. Specifically for the Trustchain two algorithms, namely NetFlow
and Temporal PageRank. Yet, while the two algorithms allow for sybil-resistant calculation of a 
metric which is related to the balance of agents. Also the accuracy of the algorithms depends on 
the amount of data that is available, making it neccessary to share data between agents in order
to better be able to estimate the probability of sybils. The sybil attack will further be discussed
in chapter 4.

\subsection{Blockwithholding attack}
In decentralized systems it can be advantageous for agents to not share some information about
their transactions that would otherwise render them in a weaker position. This is not possible
in centralized systems because users do not keep their own data which instead is stored on the
central server. Thus it is not the user's decision to share or not share information with others.

In common blockchain fabrics all information is shared with everyone and only information that is 
accepted by everyone is true. By removing the global consensus this guarantee is no longer intact.
If user's own their data, they can decide to share it or not. Agents can claim that information was
lost during transactions or that a transaction did not take place.

\subsection{Dishonest behaviour}
Some application types may require agents to act according to a specific set of rules. For example
in the Tribler application, if an agent (responder) receives two requests for contribution the 
agent should contribute to the one agent that has contributed the most in the past as that agent 
deserves to be rewarded for those past contributions. Without global consensus the agent determines
the ``goodness'' of the requesters on the basis of an unobserved information set, which is a subset
of the global network information. However the agent can also decide to not stick to the rules and
contribute to the lesser of the two requesters. Without consensus on the information set on the 
basis of which the responder decides, this dishonest behaviour cannot be detected and punished by
other agents.

\section{Research question}
From the above discussion it becomes clear that removing the global consensus from blockchain
farbics opens the system to many forms of attacks. The missing guarantees on information makes it
hard to check the correct behaviour of other agents. This makes sharing of information and 
validation of transactions an essential building block of a blockchain system without global
consensus. Yet, the question is how to enforce dissemination of transaction records without a
trusted third party. Also which information is neccessary to distribute accross the network and how
can we make sure that validation of that information is done by all nodes. Formally we can define 
the following research question:

\begin{center}
    \textit{How can we design a scalable, decentralized accounting system that ensures the distribution,
    validation and honest usage of transaction records?}
\end{center}

 
