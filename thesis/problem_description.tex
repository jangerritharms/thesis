\chapter{Problem description}
% The problem definition is the basis for the problem statement, research questions and dissertation objective.

% The general problem area.
% e.g. "The effect of nutrition on athletic performance in children is an underexplored topic."
% A globally accepted trust distribution has ever been a futuristic dream of a world without third parties and secure world-wide peer to peer transactions between unknown actors. However, a scalable solution in a deployed, production-grade software, has yet to be presented. 

% Problem definition
%   * context
%   * background
%   * specificity 
%   * relevance

% Problem statement
% It defines what the problem is.
% It explains where the problem is occurring.
% It focuses on a single problem.
% It is clearly and explicitly written.
% It is relevant to the field (or the client, if applicable).


% In bitcoin we use proof of transactions and an uncheatable consus mechanism to avoid the problem of having trust in the interacting party. However this is very costly. In real-life we don't have complete control, we always need to trust someone. If we buy food, we first give money and expect to get the food in return. 

% Everyone node has a local Temporal PageRank ranking of nodes around itself. Each node stores this ranking periodically on its hashchain. Now we need to find a way to combine multiple trust rankings into a large ranking. 
% Then, next we can define a consensus algorithm which 

% Right now we have a system to determine our personal view of the reputation in a sybil-resistant way with Temporal PageRank. However a complete reputation system needs to at least fullfil these requirements: [Reputation Systems: Facilitating Trust in Internet Interactions.]
% * Entities are long-lived, so that there is an expectation of future interaction
% * Feedback about current interactions is captured and distributed. Such information must be visible in the future.

% We can use the taxonomy on reputation systems to compare the current implementation with the one that we require in the future. Also identify the work that has been done on systems with multiple dimensions of reputation. We would like to end up with a reputation layer that is application agnostic however trust is context dependent. One of the largest problems will be how to provide a reputation fabric that is generic enough to be able to combine multiple dimensions of reputation to a score that is application agnostic. 

% We don't distribute the personal trust rankings and therefore we cannot establish the trustworthiness of an unkown node on the system. Also the reputation system is too much simplified in order to create an application agnostic reputation system.

Any reputation system requires a few basic properties in order to be useful for 
building trust among strangers. According to Resnick et al.~\cite{resnick2000reputation} those are:

\begin{itemize}
\item long-lived entities
\item feedback about current interactions is captured and distributed.
\item past feedback guides decisions
\end{itemize}

Reputation systems have been shown to work well on online centralized marketplaces such
as Ebay, Amazon or home sharing service Airbnb. The above properties are quite 
easy to fulfill in a centralized system. Long-lived entities can be enforced
through requiring document checks on sign-up, requiring real names and invalidating
email adresses after use. Feedback of interactions can be stored in databases, 
distributions is as simple as connecting a rest-api to the database. Obviously
users will want to use feedback to guide their decisions. In a decentralized 
system, users keep track of their own transactions, sharing these transactions
is their own decisions. 

In previous work we have shown how to create a multi hashchain based transaction
storage. The manipulation resistance of such a system depends mainly on the 
availability of data. In the theoretical case that all agents have all transaction
data of the network, all attacks would be detectable. This case is impossible
to achieve in a distributed system, but with a good dissemination strategy this
optimal state can be approximated. 

The distribution of data has until now been up to the agent itself. 
No policies have yet been defined on how to distribute
the transaction data and no incentive was offered to share the data. Given that
feedback on the interactions of the past are the only way to decide between 
trustworthy and untrustworthy agents, it is of high importance to have a good
dissemination mechanism in place.

\begin{center}
\textit{How to define a dissemination mechanism in order to make sharing of 
interactions outcomes strategy-proof and make attacks detectable?}
\end{center}

In order to answer this question we consider the Tribler system, a bittorrent client developed
by the Blockchain lab, which implements the TurstChain architecture. Each transaction between two 
clients in Tribler creates a block on the hashchains of both participating nodes, storing the 
amount uploaded and downloaded from.  

\paragraph{Attacks that we are trying to make detectable.}
We will look at the most important attacks on distributed reputation systems 
that are known at this point. 

\begin{itemize}
  \item block-withholding
  \item double-spend
  \item sybil
  \item self-promoting
\end{itemize}

\paragraph{A mechanism is strategy-proof} if under no circumstances can an agent
attain an advantage by not sharing information.
