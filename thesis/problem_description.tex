\chapter{Problem description}
\label{chap:problem}
\paragraph{Abstract.} Our audacious ambition is to design and create a layer of reputation on top of the core infrastructure of the internet that enables application agnostic trustful relationships between relative strangers. Such a global reputation systems needs to be distributed, scalable, tamper-proof, robust against strategic manipulation and misbehavior. Reputation system require public visibility of reputations however the state of large distributed systems are inherently unobservable. Together with a general subjectivity due to the nature of the concept of trust this creates a discrepancy between agents perception of the trustworthiness of their peers. The main question underlying this work is how diminish this discrepancy. 
locality?; trust vector or number; reputation or trust system?; automated trust system; quantifiable?;  


% 1. As discussed in the introduction in order to facilitate the digital economy in the future an open, distributed reputation system is required. The reputation system will enable trustful relationships between strangers and will need to be at the core of the internet itself.  
In the introduction we have made a case for our audacious ambition to design and create a layer 
of reputation on top of the core infrastructure of the internet that enables application agnostic 
trustful relationships between relative strangers. This requires a distributed, scalable reputation
system. Reputation systems appear in many forms but we are concerned with their digital form as only
digital networks can reach global scale with fast information distribution. Reputation systems have
previously been formally defined to include at least three components \cite{resnick2000}:

\begin{itemize}
    \item Long-lived identities
    \item Recording and distribution of feedback about interactions
    \item Use of feedback to guide interactions
\end{itemize}

We need entities to be identifiable and in existence for a long time to ensure that future interactions
between known entities are likely. If changing of identities is easy, a bad reputation is easily 
discarded and exchanged for a clean slate. We need to capture and distribute feedback of interacitons
such that entities are aware of the history and reputation of other entities on the system. Finally,
users should actually make use of the feedback on not just ignore it.

Next to the requirements of reputation systems we also have requirements for specific usecase of a
global reputation system without centralized institutations.

% 2. The reputation system we define needs to have some properties:

\begin{itemize}
    \item distributed: no entity should be owner of the reputation of all people, no single point of
    failure should exist
    \item scalable: future applications similar to those that exist today with centralized reputation
    systems should be able to handle billions of users.
    \item robust against strategic manipulation: once reputation increases in worth users of the 
    system will try to exploit the system by attacking it, alone or by colluding and the
    architecture needs to be robust against such attacks
\end{itemize}
  
This introduces additional challenges: enforcing long-lived identities is even harder without the 
assumption of a trusted central entity that can check the validity of new entities. Also creating
a central distributed record with synchronization at scale is a topic of ongoing research and 
generally seen as an unsolved problem. Finally, reputation system in general and distributed systems
specifically are intrinsically weak in protection against malicious behavior, although they are 
robust in terms of complete failures as no single point of failure exists.

As of today, no single algorithm or architecture can provide a solution that conforms with all these 
requirements. Only by combining multiple components and iterating their design can we approach a 
reputation layer that is able to conform with the requirements. This layer needs to combine these
components:

\paragraph{Identity.} At the lowest level there is the identity layer that ensures an entity is 
identifiable for other entities in the network. The most basic version of this is a simple public-
private key pair for signing and encrypting data. But creating a new key pair is cheap, therefore 
this is not enough and in a later iteration of this identity system digital entities will need to be 
bound to real-world, verified entites like government-issued passports or biometric identifiers.

\paragraph{Communication.} The internet creates a global communication network with high connectivity
across the world but it is currently not in a state that direct communication between devices is 
straight-forward. The large increase in connected devices expended the address space of IPv4 and IPv6
transition has been slow, thus network address translation creates subnetworks with local adress 
spaces. Connection from such a subspace to a server with a public adress is still simple like it is
the case with most client-server applcations on the internet, but direct communication when both 
devices are behind NATs is still not standardized. Also new routing solutions are required to ensure
communication based on the actual identity layer mentioned before instead of the IPv4 and DNS identity
layer.

\paragraph{Record and distribution mechanism} Reputation is based on feedback on interactions. 
This feedback needs to be recorded and distributed, such that other entities in the
network have a chance to respond to the history of feedback of their peers. Without a central entity
which is aware of all transactions, each node will record some transactions. It is a challenge to 
create a global record which is correct, tamper-proof and well distributed across the network.

\paragraph{Interpretation of records} Based on the recorded feedback each agent can interpret them
to form an opinion about other nodes. For a reputation system the records are seen as positive or
negative behavior and each agent can output a ranking of reputations for all peers this agent knows
of. Those rankings are calculated based on a reputation function. In the past our research group has
analyzed different reputation functions like NetFlow~\cite{OTTE2017}, 
MaxFlow~\cite{meulpolder2009bartercast} and PageRank~\cite{page1999pagerank}.

\paragraph{Application layer} We imagine that the reputation system we are developing can be used 
for any type of application that requires two entities to trust each other. This reputation layer 
will be accessible for anyone however no application will be able to delete data or lock data into
their proprietary platform. 

%     \item identity layer: digital identities are related to real-world entities and can either be real-people or entities like institutions or companies. The most simple first solution is just to identify entities by their public-private key pair.
%     * encounter record and distribution layer: encounters between digital identities needs to be recorded and distributed. Encounters are the basis for the reputation of entities, but reputation is only useful if it is gossiped such that other people can make decisions on the knowledge they have.
%     * interpretation layer: based on the known history of encounters, a reputation can be given to the known entities and decisions can be based on this. This layer contains the reputation function: NetFlow, MaxFlow, PageRank etc.

Each layer adds another level of protection: if identities are expensive and hard to create, fake 
identities will be less easy to create, protecting the system against spamming. Creating an immutable
record and distributing that information to everyone makes knowledge tamper-proof, unchangeable and 
forever, making all information reliable. On the interpretation layer additional securities can be 
enabled: we envision a concept of locality to secure against distributed attacks with global 
collaborations of malicious nodes - if we only trust agents with a certain level of latency 
attackers can only choose from nodes in the vicinity and supply of those nodes is limited. 

The concept of trust is however vague. In the analog world, trust is more of a feeling than a 
rational calculation, yet computing systems are deterministic, precise and rational. The vagueness 
partly stems from the subjective interpretation of actions which in the digital system is the 
reputation function used, but also from the difference in what each person knows about another one,
or in digitally speaking subsets of information that each one agent has. 
Yet, from a practical point of view it would be desirable if people could agree on the reputation 
and trustworthiness of agents, because it makes actions predictable and gives reputation its value. 

To see this we should go back to the discussion of indirect reciprocity, which is one of the five
forms of fostering cooperation which were introduced in chapter~\ref{chap:introduction}. People act
prosocially at a personal cost in order to build an altruistic reputation which is rewarded with 
third-parties acting prosocially towards them. Reputation is valuable because people with higher 
reputation can expect more cooperation in future interactions. This indirect reciprocity mechanism 
works as long as agents agree on what is good and what is bad reputation. Once there is ambiguity 
about the reputation of agents this value decreases as even people with a bad reputation could be 
seen as good people by others due to that ambiguity. Also only if reputation has actual value to 
agents, we can ensure prosocial behavior in the network. Therefore we need agreement on the 
reputation of agents. 

Agreement on the interpretation layer can be achieved by defining a function for all agents to use
which calculates a quantitative reputation from the history of feedback. Usually this history is 
public, visible to everyone, however this is difficult to achieve in a distributed system. Here the
second problem comes into play. The information a network node acts upon 
is a different subset of complete information on the network for each agent; each agent is 
in a different state. This situation is undesirable but inherent to systems with the requirements 
stated in the previous section. Thus, a first step towards agreeing on the reputation of agents is 
if agents agree on which data should be used as an input to the reputation function. In 
other words we have to make sure that agents disseminate their knowledge and obtain knowledge from 
other agents such that information is well distributed and available.

However, in most contexts sharing and obtaining information comes at a cost which is not negligible.
Thus agents may be reluctant to spend resources without any direct reward. There is an obvious 
network effect to agents knowing more about their peers but agents can also gain reputation by 
cooperating with agents with low reputation. Thus there is no incentive to obtain a better view of
the network.

The question that we are trying to answer is therefore:
\begin{center}
    \textit{How can we design a scalable, distributed feedback recording and distribution mechanism
    that makes dissemination and verification of transaction records incentive compatible?}
\end{center}

