\chapter{Information dissemination}

\paragraph{The strength of a reputation system} largely pivots on the availability
of data. Reputation is built through a history of interactions, but only those
that know about the history can estimate the true nature of the agent. Also, 
unfair actions can only be detected if the information about those actions is 
widespread. In centralized reputation systems, this availability in guaranteed,
as long as the central server with all data is available and not manipulated.
In decentralized systems this guarantee does not exist, availability of data
depends on the willingness of agents to share their private data.

Therefore our goal is to create a mechanism that gives agents an incentive to
share their private data. More specifically, this dissemination mechanism should
make the sharing of private data strategyproof, that is, sharing all private
data should never be less advantageous than not sharing. 

\section{Pairwise auditing}

agents group in pairs and assign a score(endorsement) to each other. The endorsement should increase
with more data shared between the two parties. There is a maximum endorsement which 
can be calculated when all data is available. The score is used as another factor
when determining the trustworthiness of an agent.

\subsection{Definition}
\subsection{Incentive design}
\paragraph{Endorsements} without any endorsements, agents are seen as not trustworthy. 
Therefore agents need to exchange data with at least a few agents in order to 
become trustworthy. Endorsements should not be accepted by default but rather
only be accepted

\paragraph{Strategy-proofness}
\subsection{Endorsement}

\section{Accusation}
\subsection{Proof}
\subsection{Untrue accusations}
