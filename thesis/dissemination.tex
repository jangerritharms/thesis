\chapter{Reputation consensus through anti-entropy}
% A specific implementation of the extension of TrustChain

% A policy that makes dissemination and verificaiton strategy-proof
In the first chapters of this thesis we introduced the incentive problem of information dissemination 
in distributed reputation systems. In the previous chapter we introduced an extension of the TrustChain
architecture that allows for agents in the network to obtain and verify each other's internal 
information state. But in order to achieve the goal of strategy-proof dissemination of information
a tangible mechanism is required that applies the advanced tools that the new architecture provides.
In this chapter we analyze one such mechanism which is based on the anti-entropy concept. We first
describe the mechanism conceptually, then discuss the implementation details and finally elaborate
on some intrinsic properties the mechanism could introduce in actual applications. In the next
chapter an implementation will be analyzed experimentally with small agent sets in order to prove
that properties from the theoretical analysis can be observed in practice.

\section{Conceptual description}
% The exact protocol: before each transaction we exchange all data and verify each other's chain

% What this allows us to do is: we can agree on a ranking that includes all of both agents data. 

% We give the receiver a chance to proof that he is worth

% We make sure that an agent has to obtain all data before making a transaction. 
Section~\ref{sec:strategy_proof_sharing} explained that the architecture itself does not solve the 
incentive problem. It rather provides the evidence on which an incentive mechanism can be built. 
Many different mechanism are possible, depending on the specific needs of the application context. 
This offers a lot of flexibility for system designers which is different from existing architectures
which a very static in their protocol and have pre-defined security and scalability properties.

\subsection{Design choice: security vs scalability}
Security, decentralization and scalability are three properties that are traded against each other 
in the design of a decentralized system. It can be argued that TrustChain was designed with scalability
as the highest priority while Bitcoin was designed with security as highest priority property. We
argue that the extension that allows for internal agent state transparency allows for flexibility in
the design choice of security and scalability. In this section we propose a mechanism that trades 
some of the scalaiblity of TrustChain for stronger security in order to show that a secure mechanism
is possible on top of the TrustChain architecture. 

\subsection{Concept: anti-entropy}
The mechanism is based on the concept of anti-entropy, which was described in \cite{demers1987epidemic}
for the purpose of maintaining mutual consistency between multiple replicas of a database. Updates
to the database can arrive at any single site and need to be forwarded to all other replicas. Demers
et. al. study three different mechanisms to disseminate the updates to other sites: direct mail, 
anti-entropy and rumor mongering.

In the direct mail mechanism, a database forwards the update to all other database immediately, which
seems like the most straight forward approach but is restricted by the fact that each database does
not know about all other databases. 

Anti-entropy is a process in which each database periodically chooses a partner database and both 
exchange all database contents in order to resolve any differences between the two. The process was
found to be reliable but slower than direct mail. 

The final mechanism is called Rumor Mongering. Sites consider updates ``hot rumors'' after receiving
them for the first time. While the site considers the rumor ``hot'', it choses periodically another
site and informs it about the rumor. When the site has encountered a certain amount of sites which
were already aware of the ``hot rumor'', that update is retained but site stops with actively
propagating the update. 

For the purpose of this work, we will focus on the concept of anti-entropy. Direct mail is not a
viable option because for large social networks the embedded social networks are a very small 
subset and the rest of the agents in the network would not be informed of updates. Rumor mongering 
effects are best observed in larger networks however this work is concerned with conceptual analysis
and the experimental analysis in the next section concerns small networks. Also there are more 
algorithms than these three but anti-entropy fits the architecture of TrustChain very well and is 
a good starting point for analyasis of dissemination mechanisms. The analysis of other mechanisms
will be subject of future work.

\subsection{Replicated databases vs TrustChain}
The context of the work of Demers et. al. is similar to the context that this work is concerned with
in many regards. Replicated databases are a distributed system as all instances of the database are
independent, equal entities, just like the agents in the TrustChain network. Each agent has an 
internal state which is equal to the set of transactions that agent is aware of which is equal to
the state of the database which is equal to the entries that database is aware of. Our goal is to 
propagate information on new transactions just like the goal of Demers et. al. was to propagate 
updates to the databases. 

In the context of reputation systems anti-entropy allows for two agents to align on their knowledge
of the social network, that is to obtain the same embedded social network and agree on the reputation
of all agents in that network.
Two agents, $a_i$ and $a_j$ have two different internal states, represented by the sets of encounters
$E_i$ and $E_j$. There can be some overlap between the two sets, but that is not guaranteed. Agent 
$a_i$ chooses to synchronize states with agent $a_j$. Both agents send their own set of known 
encounters and merge them. This results in a new set $E_{i,j} = E_i \bigcup E_j$. In the context 
of TrustChain this translates to the exchange of transaction blocks, such that after the exchange 
both agents have the exact same set of transaction blocks. As the reputation of peers is calculated
from the set of transactions both agents agree on a single reputation vector. If the two agents also
use the same function for trust calculation both can even agree on a single trust vecotr. The two 
agents have reached consensus on trust and reputation. 

The exchange of information will be recorded in the form of exchange blocks on the chain of both 
participating agents as explained in the previous chapter, section~\ref{sec:implementation_state_transparency}.

The consensus is only reached at one point in time and is not maintained. Once any of the two agents
conducts another anti-entropy exchange or a transaction, the other agent is not required to be 
informed or to observe the interaction. That is after the exchange both internal states can diverge
until the same two agents happen to perform another state synchronization. 

In the work of Demers et. al. database instances chose partners for anti-entropy exchanges at random
which is a valid strategy as each peers updates seem equally important. In contrast, reputation 
systems should value the information about possible future interaction partners as more important.
Therefore, our proposed mechanism requires agents to at least perform an anti-entropy exchange with
those agents that they will have an interaction next. That way, both parties of a transaction are 
required to obtain and verify each others information in order to make sure that the transaction 
will be done on top of a valid state. If any party does not agree with the state of the opposite 
party, the transaction will not take place. If any party performs a transaction eventhough the 
information clearly shows misconduct on the part of the partner, they will also be held responsible 
for not performing their validation responsibility. Without the requirement of validating transaction
partners, agents can purposefully exchange data with honest agents but perform interactions with 
dishonest agents and later claim to have had no knowledge of the dishonesty of the partners. 

\section{Implementation of anti-entropy exchanges}
In the previous section we described the anti-entropy method for information exchanges between 
agents. This section elaborates on the implementation details of the mechanism in the TrustChain 
architecture. 



\section{Advanced implicaitons of anti-entropy}
Additional considerations: 

locality

strategy proofness 

